\documentclass{article}
\usepackage[T1]{fontenc}                
\usepackage[utf8]{inputenc} 
\usepackage[ngerman]{babel} 
\usepackage{enumerate}
\usepackage{geometry}
\usepackage{titlesec}
\usepackage{hyperref}
\usepackage{ifthen}
\usepackage{color}
\usepackage{helvet}
\renewcommand{\familydefault}{\sfdefault}
% Disable hyphenation for better copy-pasteability
\tolerance=1
\emergencystretch=\maxdimen
\hyphenpenalty=10000
\hbadness=10000
% Disable page numbers for better copy-pasteability
\pagenumbering{gobble}

\newboolean{publish}

%=================== VARIABLEN ========================
\newcommand{\fachschaft}{Musterfachschaft}
\newcommand{\beschlussgremium}{FSV} %FSV oder FSVV
\newcommand{\beschlussdatum}{01.01.2020} %DD.MM.YYYY
\newcommand{\vorsitz}{Max Mustermann} %Vorsitzender
\setboolean{publish}{false} %false - Beschlussvorlage, true - Veröffentlichung
\newcommand{\fsrgroesse}{neun}
%======================================================



\setlength\parindent{0pt}
\geometry {a4paper, top= 25mm, bottom=25mm, left=25mm, right=25mm}
\titleformat{\part}{\fontsize{15pt}{15pt}\bfseries}{\thepart .\ }{0pt}{}
\titleformat{\chapter}{\fontsize{15pt}{15pt}\bfseries}{\thechapter .\ }{0pt}{}
\titleformat{\section}{\fontsize{11pt}{13pt}\bfseries}{\S \ \thesection \ }{0pt}{\normalsize}

\begin{document}
\noindent
\begin{center}
    \huge \textbf{Satzung der Fachschaft \linebreak[1] \fachschaft} \\
    \large \ifthenelse{\boolean{publish}}{}{\vspace{2ex} \textcolor{red}{\textbf{- Beschlussvorlage -}}}
\end{center}

\section*{Präambel}
\noindent
Als Teil der Studierendenschaft der Rheinischen Friedrich-Wilhelms-Universität Bonn und in Ausübung ihres Rechts auf Selbstverwaltung hat sich die Fachschaft \fachschaft\ die
folgende Satzung gegeben.
 

\part{Fachschaft}
\section{Begriffsbestimmung und Rechtsstellung}
\begin{enumerate}[(1)]
\item Die Fachschaft \fachschaft , nachfolgend bezeichnet als „Fachschaft“, bilden alle Studierenden, die in den der Fachschaft \fachschaft\ zugeordneten Studienfächern im Hauptfach eingeschrieben sind. Die Zuordnung erfolgt gemäß der Anlage „Fachschaftenliste“ zur Geschäftsordnung der Fachschaftenkonferenz (FKGO) (vgl. §~22 SdS).
\item Die Fachschaft vertritt die spezifischen Interessen ihrer Mitglieder. Sie vertritt darüber hinaus im Rahmen ihrer Möglichkeiten Belange von Studierenden, die an einem Studienangebot eines Faches teilnehmen, das der Fachschaft zugeordnet ist, auch wenn diese Studierende nicht Mitglieder der Fachschaft sind.
\end{enumerate}

\section{Organe der Fachschaft}
\begin{enumerate}[(1)]
\item Die Fachschaft äußert ihren Willen durch ihre Organe und deren Wahl.
\item Organe der Fachschaft sind:
\begin{enumerate}[1.]
    \item die Fachschaftsvollversammlung (FSVV)
    \item die Fachschaftsvertretung (FSV)
    \item der Fachschaftsrat (FSR)
\end{enumerate}
\item Die Amtszeit der Mitglieder der gewählten Organe beträgt ein Jahr. Bis zur Neuwahl der Nachfolgemitglieder bleiben die Mitglieder der betreffenden Organe kommissarisch im Amt.
\end{enumerate}

\section{Gemeinsame Aufgaben der Organe FSV und FSR}
\begin{enumerate}[(1)]
\item Die Fachschaft fördert auf der Grundlage der verfassungsmäßigen Ordnung die politische Bildung und das staatsbürgerliche Verantwortungsbewusstsein der Mitglieder der Fachschaft.
\item Die Organe FSV und FSR wirken an der fachlichen und organisatorischen Gestaltung des Studiums mit und vertreten die Studierenden ihrer Fachbereiche gegenüber der Professorenschaft, den Gremien der Universität und den übrigen Gremien der Studierendenschaft.
\item Die Organe FSV und FSR vertreten die hochschulpolitischen Belange der Fachschaft und beziehen Stellung zu hochschulpolitischen Fragen. Eine über die Aufgaben der Organe FSV und FSR hinausgehende allgemeinpolitische Willensbildung vollzieht sich in den studentischen Vereinigungen der Hochschule.
\end{enumerate}

\part{Die Fachschaftsvertretung (FSV)}
\section{Rechtsstellung der FSV}
Die FSV ist Beschlussorgan der Fachschaft. (vgl. §~27 Absatz 3 SdS)

\section{Zusammensetzung und Zusammentritt der FSV}
\begin{enumerate}[(1)]
\item Die Zahl der Mitglieder der FSV richtet sich nach den Regelungen der Satzung der Studierendenschaft (SdS) und der Fachschaftswahlordnung (FSWO).
\item Die Einladung zu einer FSV-Sitzung erfolgt in Textform. Sie muss mindestens sieben Tage vor der geplanten Sitzung an alle FSR- und FSV-Mitglieder verschickt werden. Zu der Sitzung muss auch öffentlich eingeladen werden.
\item Die Mitglieder der FSV sind grundsätzlich verpflichtet, an den Sitzungen teilzunehmen, sofern sie nicht begründet entschuldigt sind.
\end{enumerate}


\section{Aufgaben und Zuständigkeit der FSV}
\begin{enumerate}[(1)]
    \item Die FSV trifft alle Entscheidungen von grundlegender oder gehobener Bedeutung für die Fachschaft, die über den regulären Geschäftsbetrieb des FSR hinausgehen.
    \item Die FSV wählt den FSR.
	\item Die FSV wählt den Kassenprüfungsausschuss.
	\item Die FSV wählt den Wahlausschuss.
	\item Die FSV beschließt über den Haushaltsplan.
	\item Die FSV beschließt mit der Mehrheit ihrer gewählten Mitglieder die politische und finanzielle Entlastung des FSR. Die finanzielle Entlastung kann nicht verweigert werden, wenn die Kassenprüfung keine Ungenauigkeiten ergibt. Die Entlastung muss von einem Mitglied der FSV beantragt werden. Finanzielle Entlastung kann auch von den Kassenprüfern beantragt werden. Auf Antrag eines Mitglieds der FSV muss eine Einzelentlastung durchgeführt werden.
\end{enumerate}

\section{Das Präsidium der FSV und seine Aufgaben}
\begin{enumerate}[(1)]
	\item Das Präsidium der FSV besteht aus
	\begin{enumerate}[1.]
		\item dem Vorsitzenden,
		\item dem stellvertretenden Vorsitzenden,
		\item dem Schriftführer.
	\end{enumerate}
    \item Alle Mitglieder des Präsidiums müssen FSV-Mitglieder sein und werden einzeln in geheimer Wahl auf der konstituierenden Sitzung gewählt.
    \item Die Ämter des Präsidiums der FSV sind unvereinbar mit der Mitgliedschaft im FSR.
	\item Zur Wahl des Präsidiums bedarf es der Mehrheit der gewählten Mitglieder der FSV. Erhält im ersten Wahlgang kein Kandidat die notwendige Stimmenzahl, so findet unverzüglich ein zweiter Wahlgang statt. Erreicht auch in diesem Wahlgang kein Kandidat die notwendige Stimmenzahl, so gilt im dritten Wahlgang der Kandidat als gewählt, der die relative Mehrheit der Stimmen auf sich vereint. Während einer Wahl mit mehreren Wahlgängen können neue Kandidaten nur für die Wahlliste vorgeschlagen werden, wenn die Mehrheit der anwesenden Mitglieder einem Antrag auf Öffnung der Wahlliste zustimmt.
	\item Der kommissarische Status des FSR-Vorsitzenden lässt eine auf einer FSV-Sitzung erfolgende Wahl ins Präsidium der FSV zu, wenn in derselben Sitzung ein Nachfolger für das Amt des FSR- Vorsitzenden gewählt wird.
	\item Tritt ein Mitglied des FSV-Präsidiums zurück, wählt die FSV unverzüglich einen Nachfolger. Kann die Wahl nicht auf derselben Sitzung erfolgen, so führt das ausgeschiedene Mitglied sein Amt kommissarisch bis zur Nachwahl weiter.
	\item Mitglieder des Präsidiums können nur mit der Mehrheit der Stimmen der gewählten FSV-Mitglieder durch die Wahl eines Nachfolgers abberufen werden.
	\item Der Schriftführer ist für die Erstellung des Sitzungsprotokolls verantwortlich. Er kann an seiner statt ein Mitglied der FSV zum Protokollanten ernennen. Der Schriftführer ist dafür verantwortlich, dass das Protokoll der FSV-Sitzung eine Woche nach der Sitzung sowohl in Schrift- als auch in digitaler Form ausgefertigt an den FSV-Vorsitzenden weitergeleitet und vom FSV-Vorsitzenden jeweils zur nächsten FSV-Sitzung allen Mitgliedern ausgehändigt wird. Dem Protokoll ist eine Anwesenheitsliste der jeweiligen FSV-Sitzung hinzuzufügen.
	\item Über die Vollständigkeit und Richtigkeit des Protokolls wird in der jeweiligen FSV-Sitzung mit der Mehrheit der anwesenden Mitglieder abgestimmt. Danach hat jedes FSV-Mitglied das Recht, eine Stellungnahme zum Protokoll abzugeben. Gleiches gilt für andere Personen, die zu einem bestimmten Punkt das Wort erhoben haben.
	\item Das beschlossene Protokoll ist der Fachschaft
    unverzüglich für mindestens sieben Tage durch Aushang und mindestens für 2 Jahre an geeigneter Stelle im Internet bekanntzugeben.
	\item Der Vorsitzende der FSV führt ihre laufenden Geschäfte. Er beruft die FSV ein, wenn
	\begin{enumerate}[1.]
	    \item der FSR-Vorsitzende
        \item die Mehrheit des FSR,
		\item sechs Mitglieder der FSV,
		\item die FSVV,
		\item fünf Prozent der Mitglieder der Fachschaft 
	\end{enumerate}
	dies unter Angabe von zu behandelnden Tagesordnungspunkten in Textform verlangen. Enthält das Verlangen keinen Sitzungstermin, so ist die FSV innerhalb von zwei Wochen einzuberufen. Die Ladungsfrist nach §~5 Absatz 3 muss eingehalten werden.
	\item Bei Abwesenheit oder sonstiger Verhinderung wird der FSV-Vorsitzende durch den stellvertretenden FSV-Vorsitzenden vertreten.
	\item Bei Abwesenheit oder sonstiger Verhinderung wird der stellvertretende FSV-Vorsitzende durch den Schriftführer vertreten.
	\item Bei Abwesenheit oder sonstiger Verhinderung wird der Schriftführer durch das älteste anwesende FSV-Mitglied vertreten.
\end{enumerate}

\section{Ausscheiden und Nachrücken von Mitgliedern}
\begin{enumerate}[(1)]
    \item Ein Mitglied scheidet aus der FSV aus
    	\begin{enumerate}[1.]
	    \item durch Niederlegung seines Amtes,
        \item durch Ausscheiden aus der Fachschaft, insbesondere durch Exmatrikulation, Umschreibung oder Tod.
	\end{enumerate}
	\item Die Wiederbesetzung eines freigewordenen Sitzes regelt die Fachschaftswahlordnung (FSWO). 
\end{enumerate}

\section{Beschlüsse der FSV}
\begin{enumerate}[(1)]
    \item Rede- und Antragsrecht haben alle Mitglieder der Fachschaft \fachschaft.
    \item Stimmrecht haben nur FSV-Mitglieder.
    \item Auf schriftlichen Antrag von mindestens drei Mitgliedern der FSV hat ein FSR-Mitglied während der den Antrag betreffenden nachfolgenden Sitzung anwesend zu sein (Zitierrecht).
    \item Die FSV gilt als ist beschlussfähig, wenn mindestens die Hälfte der FSV-Mitglieder anwesend ist.
    \item Die FSV gilt solange als beschlussfähig, bis auf Antrag eines FSV-Mitgliedes durch die Sitzungsleitung das Gegenteil festgestellt wird.
	\item Die Beschlussfähigkeit wird auf Antrag unverzüglich festgestellt. Sie ist gegeben, wenn mehr als die Hälfte der FSV-Mitglieder anwesend ist. Ein Einspruch gegen diesen Antrag ist nicht möglich. Der FSV- Vorsitzende überprüft die Beschlussfähigkeit durch namentlichen Aufruf.
	\item Bei Beschlussunfähigkeit muss innerhalb von 10 Tagen eine zweite Sitzung mit der gleichen Tagesordnung einberufen werden. Die normalen Ladungsfristen sind zu wahren. Diese Sitzung ist unabhängig von der Zahl der anwesenden Mitglieder beschlussfähig. Die Einladung hat ausdrücklich auf diesen Umstand hinzuweisen.
	\item Ein Beschluss ist rechtmäßig zustande gekommen, wenn
	\begin{enumerate}[1.]
		\item Die Sitzung der FSV fristgerecht einberufen wurde,
		\item die FSV beschlussfähig war und
		\item er die einfache Mehrheit gefunden hat, soweit diese Satzung oder eine höhere Rechtsquelle nichts anderes vorschreibt.
	\end{enumerate}
	\item FSV-Beschlüsse der laufenden Sitzungsperiode können durch Beschluss mit einer Mehrheit von zwei Dritteln der gewählten FSV-Mitglieder aufgehoben werden.
\end{enumerate}

\section{Ausschüsse der FSV}
\begin{enumerate}[(1)]
    \item Die FSV wählt die Mitglieder des Wahlausschusses sowie den Wahlleiter mit der Mehrheit ihrer gewählten Mitglieder. Näheres regelt die Fachschaftswahlordnung (FSWO).
    \item Die FSV wählt als Mitglieder des Kassenprüfungsausschusses mindestens zwei Kassenprüfer mit der Mehrheit ihrer gewählten Mitglieder.
    \item Kassenprüfer kann nur sein, wer weder im geprüften Zeitraum noch zum Prüfungszeitpunkt Mitglied des FSR oder Teil des Vorstands der FSV war beziehungsweise ist. Kassenprüfer müssen Teil der Studierendenschaft der RFWU Bonn sein.
    \item Die Kassenprüfer kontrollieren die ordnungsgemäße Kassenführung des Haushaltsjahres für dessen Kontrolle sie gewählt wurden und erstatten der FSV über das Ergebnis der Prüfung Bericht.
\end{enumerate}

\part{Der Fachschaftsrat (FSR)}
\section{Rechtsstellung des FSR}
\begin{enumerate}[(1)]
    \item Der FSR repräsentiert und vertritt die Fachschaft und führt ihre Geschäfte. Der FSR ist im Rahmen der zu besorgenden Geschäfte sowie im Eilfall auch Beschlussorgan, im Übrigen führt er die Beschlüsse der FSV aus.
    \item Der FSR-Vorsitzende hat Beschlüsse, Unterlassungen oder Maßnahmen der FSV, des FSR, sowie der FSVV sofern sie gegen geltendes Recht verstoßen, gegenüber dem Vorsitzenden der Fachschaftenkonferenz (FK) zu beanstanden.
\end{enumerate}

\section{Zusammensetzung des FSR}
\begin{enumerate}[(1)]
    \item Der FSR besteht aus \fsrgroesse\ regulären Mitgliedern.
    \item Falls die FSV nach §~27 Absatz 5 SdS zusätzliche Referentinnen in den FSR wählt, vergrößert sich die Zahl der FSR-Mitglieder entsprechend.
    \item Der FSR-Vorstand besteht aus:
	\begin{enumerate}[1.]
		\item dem Vorsitzenden,
		\item dem stellvertretenden Vorsitzenden,
		\item und dem Finanzreferenten
	\end{enumerate}
	\item Die Vorstandsmitglieder können sich bei gegenseitigem Einvernehmen wechselseitig vertreten.
\end{enumerate}

\section{Sitzungen des FSR}
\begin{enumerate}[(1)]
    \item Der FSR tritt in öffentlicher Sitzung zusammen:
    \begin{enumerate}[1.]
        \item während der Vorlesungszeit grundsätzlich einmal wöchentlich
        \item auf eigenen Beschluss,
        \item auf Beschluss der FSV.
    \end{enumerate}
    \item FSR-Sitzungen werden der Fachschaft durch öffentliche schriftliche Ankündigung, den FSR-Mitgliedern zusätzlich in Textform mindestens zwei Tage im voraus bekanntgegeben.
    \item Die Mitglieder des FSR sind grundsätzlich verpflichtet, an den Sitzungen teilzunehmen, sofern sie nicht begründet entschuldigt sind.
    \item Der FSR ist verpflichtet, während der Sitzungen Protokoll zu führen.
    \item Sofern er sich keine eigene Geschäftsordnung gibt, gilt für den FSR die Geschäftsordnung der Fachschaftenkonferenz, soweit anwendbar, entsprechend.
\end{enumerate}

\section{Wahl des FSR}    
\begin{enumerate}[(1)]
    \item Der FSR-Vorstand wird von der FSV mit der Mehrheit ihrer gewählten Mitglieder gewählt.
    \item Die weiteren Mitglieder des FSR neben dem Vorstand werden durch den FSR-Vorsitzenden vorgeschlagen und, auf Verlangen einzeln, mit der Mehrheit der gewählten Mitglieder der FSV gewählt.
    \item Der zu wählende FSR-Vorsitzende muss der FSV zum Zeitpunkt seiner Wahl angehören.
    \item Alle Mitglieder des FSR müssen Mitglieder der Fachschaft sein.
    \item Die Mitgliedschaft im FSR ist unvereinbar mit Ämtern des Präsidiums der FSV und der Mitgliedschaft im Kassenprüfungsausschuss.
    \item Mitglieder des FSR-Vorstandes können nur mit der Mehrheit der Stimmen der gewählten FSV-Mitglieder durch die Wahl eines Nachfolgers abberufen werden. Alle anderen Mitglieder des FSR können mit der Mehrheit der Stimmen der gewählten FSV-Mitglieder abberufen werden.
    \item FSR-Mitglieder können jederzeit zurücktreten. Sie sind jedoch verpflichtet, die Geschäfte bis zur Wahl eines Nachfolgers weiterzuführen. Tritt ein Mitglied des FSR-Vorstandes zurück, wählt die FSV unverzüglich einen Nachfolger.
    \item Ein Mitglied scheidet aus der FSR aus
    \begin{enumerate}[1.]
    	\item durch Abberufung,
	    \item durch Niederlegung seines Amtes,
        \item durch Ausscheiden aus der Fachschaft, insbesondere durch Exmatrikulation, Umschreibung oder Tod.
	\end{enumerate} 
\end{enumerate}

\section{Beschlüsse des FSR}
\begin{enumerate}[(1)]
    \item Rede- und Antragsrecht haben alle Mitglieder der Fachschaft \fachschaft.
	\item Stimmrecht haben nur FSR-Mitglieder.
	\item Ein Beschluss ist rechtmäßig zustande gekommen, wenn
	\begin{enumerate}[1.]
		\item der FSR beschlussfähig war und
		\item er die relative Mehrheit gefunden hat, soweit die Satzung nichts anderes vorschreibt.
	\end{enumerate}
	\item Der FSR gilt solange als beschlussfähig, bis auf Antrag eines FSR-Mitgliedes durch den Vorsitzenden das Gegenteil festgestellt wird.
	\item Die Beschlussfähigkeit wird auf Antrag unverzüglich festgestellt. Sie ist gegeben, wenn mehr als die Hälfte der FSR-Mitglieder anwesend ist. Ein Einspruch gegen diesen Antrag ist nicht möglich. Der FSR-Vorsitzende überprüft die Beschlussfähigkeit durch namentlichen Aufruf.
	\item Bei Beschlussunfähigkeit muss nach spätestens 14 Tagen eine zweite Sitzung mit der gleichen Tagesordnung einberufen werden. Die normalen Ladungsfristen sind zu wahren. Diese Sitzung ist unabhängig von der Zahl der anwesenden Mitglieder beschlussfähig. Die Einladung hat ausdrücklich auf diesen Umstand hinzuweisen.
    \item FSR-Beschlüsse der laufenden Sitzungsperiode können durch Beschluss mit einer Zweidrittelmehrheit der gewählten Mitglieder des FSR oder einer einfachen Mehrheit der FSV aufgehoben werden.
\end{enumerate}

\section{Aufgaben und Zuständigkeiten des FSR}
\begin{enumerate}[(1)]
    \item Der FSR kann durch Mehrheitsbeschluss Aufgabengebiete an einzelne FSR-Mitglied vergeben.
    \item Der FSR-Vorsitzende bestimmt die Richtlinien der Arbeit des FSR und trägt dafür die Verantwortung. Innerhalb dieser Richtlinien ist jedes FSR-Mitglied gegenüber dem Vorsitzenden für sein Aufgabengebiet verantwortlich.
	\item Der FSR-Vorsitzende ist insbesondere dafür verantwortlich, die Arbeit der Organe der Fachschaft an alle Mitglieder der Fachschaft zu kommunizieren.
	\item Der FSR-Vorsitzende hat Beschlüsse, Unterlassungen oder Maßnahmen der FSV, des FSR und der FSVV zu beanstanden, sofern sie gegen geltendes Recht verstoßen.
\end{enumerate}

\part{Die Fachschaftsvollversammlung (FSVV)}
\section{Rechtsstellung der FSVV}
\begin{enumerate}[(1)]
    \item Die FSVV ist die Versammlung der Mitglieder der Fachschaft \fachschaft.
    \item Die FSVV ist oberstes Beschlussorgan und dient der Information ihrer Mitglieder. Soweit keine FSV besteht, übernimmt sie die Aufgaben der FSV, sofern durch diese Satzung, die Fachschaftswahlordnung oder eine höherere Rechtsquelle keine andere Zuständigkeit geregelt ist.
\end{enumerate}

\section{Einberufung und Durchführung der FSVV}
\begin{enumerate}[(1)]
	\item Der Vorsitzende des FSR beruft die FSVV ein:
	\begin{enumerate}[1.]
		\item auf Beschluss der FSV
		\item auf Beschluss des FSR
		\item auf schriftlichen Antrag von mindestens 5\% der Mitglieder der Fachschaft, sofern der Antrag eine Tagesordnung enthält.
	\end{enumerate}
	\item Die Ankündigung der FSVV erfolgt mindestens eine Woche vor ihrer Durchführung in Textform. Die Ankündigung enthält mindestens
	\begin{enumerate}[1.]
		\item die genaue Zeit und Ortsangabe der FSVV sowie
		\item ihre Tagesordnung
	\end{enumerate}
	\item Die FSVV wählt zu Beginn jeder Versammlung einen Versammlungsleiter.
	\item Für die FSVV gilt die Geschäftsordnung der Fachschaftenkonferenz (FKGO) soweit anwendbar, falls sie sich keine eigene Geschäftsordnung gibt.
\end{enumerate}

\section{Aufgaben und Zuständigkeiten der FSVV}
\begin{enumerate}[(1)]
    \item Die FSVV ist oberstes Beschlussorgan der Fachschaft und dient der Information ihrer Mitglieder.
	\item Soweit keine FSV besteht, übernimmt die FSVV die Aufgaben der FSV, sofern durch Ordnung oder Satzung keine andere Zuständigkeit geregelt ist.
\end{enumerate}

\section{Beschlüsse der FSVV}

\begin{enumerate}[(1)]
	\item Rede-, Stimm- und Antragsrecht haben alle Mitglieder der Fachschaft.
	\item Die Entscheidungen der FSVV binden alle Organe der Fachschaft. 
	\item Beschlüsse der FSVV können nur durch eine weitere FSVV mit der entsprechenden Mehrheit aufgehoben werden.
	\item Ein Beschluss ist rechtmäßig zustande gekommen, wenn
	\begin{enumerate}[1.]
		\item die FSVV beschlussfähig war und
		\item er die einfache Mehrheit gefunden hat.
	\end{enumerate}
	\item Bei Beschlussunfähigkeit muss nach spätestens 14 Tagen eine zweite Sitzung mit der gleichen Tagesordnung einberufen werden. Die normalen Ladungsfristen sind zu wahren. Die Einladung hat ausdrücklich darauf hinzuweisen, dass diese Sitzung unabhängig von der Zahl der anwesenden Mitglieder beschlussfähig ist.
\end{enumerate}

\part{Haushalts- und Wirtschaftsführung}
\section{Grundsätze der Haushaltsführung}
\begin{enumerate}[(1)]
    \item Die Haushalts- und Wirtschaftsführung richtet sich nach den Vorgaben der Satzung der Studierendenschaft und der HWVO NRW.
    \item Das Haushaltsjahr der Fachschaft beginnt am 1. April eines jeden Jahres und endet am 31. März des jeweiligen Folgejahres.
    \item Bei der Aufstellung und Ausführung des Haushaltsplans sind die Grundsätze der Wirtschaftlichkeit und Sparsamkeit zu beachten.
    \item Dem Abschluss von Verträgen über Lieferungen und Leistungen muss ein Preisvergleich vorausgehen. Bei Aufträgen mit einem Wert von mehr als 1.000 Euro sind mindestens 3 Angebote im Wettbewerb einzuholen, bei Aufträgen mit einem Wert von mehr als 10.000 Euro sind mindestens 6 Bewerber/innen zur Angebotsabgabe aufzufordern. Der Preisvergleich ist aktenkundig zu machen und die Vergabeentscheidung zu dokumentieren.
    \item Dem Finanzreferenten obliegt die Finanzführung der Fachschaft. Er führt über alle Einnahmen und Ausgaben der Fachschaft ordnungsgemäß Buch.
\end{enumerate}

\section{Haushaltsplan}
\begin{enumerate}[(1)]
    \item Der Finanzreferent hat vor Beginn des Haushaltsjahres einen ausgeglichenen Haushaltsplan aufzustellen und diesen der FSV in drei Lesungen auf mindestens zwei getrennten Sitzungen vor Beginn des Haushaltsjahres zur Abstimmung vorzulegen, wobei die zweite und dritte Lesung in der gleichen Sitzung stattfinden dürfen.
    \item Anschaffungen und Ausgaben, die von den im Haushaltsplan unter einem flexiblen Titel, etwa \glqq Sonstiges\grqq, ausgewiesenen Geldern getätigt werden und die einen Höchstbetrag von 200 Euro überschreiten, sind vor der Anweisung von der FSV gesondert zu beschließen.
    \item Überplanmäßige oder außerplanmäßige Ausgaben sind vor Inkrafttreten eines Nachtrags zum Haushaltsplan, der sie vorsieht, nur dann zulässig, wenn sie unabweisbar sind. Sie sind der FSV unverzüglich anzuzeigen. Nachträge zum Haushaltsplan können nur für das laufende Haushaltsjahr eingebracht werden.
\end{enumerate}

\section{Ausgabevollmacht}
Zur finanziellen Verpflichtung der Fachschaft sind die Unterschriften des FSR-Vorsitzendes und des Finanzreferenten oder die Unterschrift des zuständigen Referenten nach Zustimmung des FSR-Vorsitzendes und des Finanzreferenten erforderlich. Der FSR kann gegen die Stimmen von Fachschaftssprecher und Finanzreferent keine finanziell erheblichen Vorhaben beschließen. Der FSR kann mit der Mehrheit der gewählten Mitglieder Ausgaben beschließen, sofern der FSR-Vorsitzende oder der Finanzreferent mit der Mehrheit stimmen.

\section{Einnahmeverpflichtung}
Der Finanzreferent ist verpflichtet, im Rahmen der Finanzordnung der Studierendenschaft für Unterstützung durch allgemeine Fachschaftengelder (AFSG) sowie die üblichen Beihilfen im Rahmen besonderer Fachschaftengelder (BFSG) zu sorgen, soweit diese nach Maßgabe der Fachschaftenkonferenz unterstützt werden.

\section{Kassenprüfung und -abschluss}
\begin{enumerate}[(1)]
    \item Die Kassenprüfer der FSV führen folgende Prüfungen durch:
    \begin{enumerate}[1.]
        \item eine Haushaltsjahresabschlussprüfung;
        \item eine Abschlussprüfung nach dem Ende der Amtszeit des FSR.
    \end{enumerate}
    Unabhängig davon wird die Kasse von den Kassenprüfern mindestens einmal jährlich unangekündigt geprüft. 
    \item Die Kassenprüfung dient dem Zweck festzustellen, ob insbesondere
    \begin{enumerate}[1.]
        \item der Kassen-Ist-Bestand mit dem Kassen-Soll-Bestand übereinstimmt,
        \item die Buchungen nach der Zeitfolge mit den Buchungen nach der im Haushaltsplan vorgesehenen Ordnung übereinstimmen und
        \item die Belege den Buchungen des Kassenbuches entsprechen.
    \end{enumerate}
    \item Über die Kassenprüfung ist Protokoll zu führen, in das die Kassen- und Kontobestände aufzunehmen sind.
\end{enumerate}

\part{Schlussbestimmungen}

\section{Einrichtung einer FSV}
\begin{enumerate}[(1)]
    \item Hat die Fachschaft weniger als 500 Mitglieder, wird anstelle einer FSV der FSR durch die Mitglieder der Fachschaft direkt gewählt.
    \item In diesem Fall finden  die Regelungen über die FSV keine Anwendung. Befugnisse und Aufgaben der FSV fallen der FSVV zu, sofern diese Satzung oder eine höhere Rechtsquelle keine andere Zuständigkeit festlegen.
\end{enumerate}

\section{Satzungsänderung}
\begin{enumerate}[(1)]
    \item Diese Satzung kann durch Beschluss einer Änderungssatzung geändert werden. Für diesen Beschluss ist eine Mehrheit von zwei Dritteln der gewählten FSV-Mitglieder oder die Mehrheit der anwesenden Fachschaftsmitglieder auf einer beschlussfähigen FSVV nötig. Die Regelungen zu außerordentlichen FSV- und FSVV- Sitzungen sind unanwendbar.
    \item Dieser Beschluss muss jedes Mal in drei Lesungen auf mindestens zwei getrennten Sitzungen gefasst werden, wobei die zweite und dritte Lesung in der gleichen Sitzung stattfinden dürfen.
    \item Der Tagesordnungspunkt „Satzungsänderung“ muss bereits in der Einladung zur betreffenden FSV-Sitzung oder FSVV-Sitzung angekündigt werden. In der Einladung müssen die zu ändernden Vorschriften ausdrücklich benannt werden. Dem Einladungsschreiben ist weiterhin der Wortlaut der beantragten Satzungsänderung beizufügen.
    \item Diese Satzung und etwaige Änderungssatzungen treten jeweils am Tag nach ihrer Veröffentlichung auf der Bekanntmachungsplattform der Studierendenschaft in Kraft. Sie sind unverzüglich der Fachschaft durch ortsüblichen Aushang und an geeigneter Stelle im Internet bekanntzugeben.
\end{enumerate}
\vspace{8ex}
\ifthenelse{\boolean{publish}}{\centering \textit{Ausgefertigt aufgrund des Beschlusses der \beschlussgremium\ der Fachschaft \fachschaft\ vom \beschlussdatum.}\\[10mm]\vorsitz \\[1mm] Vorsitz der \beschlussgremium}{\centering \textcolor{red}{- \textbf{Beschlussvorlage} - \\[2mm] Nach Beschluss ist dieses Dokument in seiner beschlossenen Form durch den Vorsitz des Beschlussorgans zu unterzeichnen und zu veröffentlichen.}}
\end{document}
